\documentclass[a4paper]{report}
\usepackage[utf8]{inputenc}
\usepackage{anysize}
\usepackage[a4paper,margin=7em]{geometry}
\usepackage{float}
\usepackage{graphicx} % Required for figures
\usepackage{booktabs}
\usepackage{hyperref}
\usepackage{fontawesome5}
\usepackage{multirow}
\usepackage{array}
\usepackage{lipsum}
\usepackage[export]{adjustbox}
\usepackage{marvosym}
\usepackage{tabularx}
%\usepackage{ltablex}
\usepackage{tabu}
\usepackage{longtable}[=v4.13]
\usepackage{enumitem}
\setitemize{itemsep=2pt, topsep=2pt, align=right, labelindent=\parindent, leftmargin=!, itemindent=!, labelsep=.5em, labelwidth=!}

\def\secsep{\hrule width5cm}
%\tabulinesep=0pt
%\extrarowsep=-3pt
%\def\secsep{\hrule}
\title{\bfseries\Huge CV}
\begin{document}
%\section*{\Huge CV}\subsection*{Thomas Krogh Lohse}
\section*{\centering \Huge CV}\subsection*{\centering Thomas Krogh Lohse}

\begin{tabularx}{\textwidth}{lX}
    \toprule%\\[-10pt]
    \href{https://maps.app.goo.gl/mtFWbUVz1f8x7saS8}{\faIcon{map-marker-alt}~Aalborg Øst 9220, Denmark}& 
    \multirow[t]{7}{=}{
        Jeg er en Software kandidatstuderende (MSc) ved Aalborg Universitet (AAU).
        Min fascination af programmering har ført mig på en rejse af selvstændig læring siden jeg var 16 år gammel.
        Mine interesser omfatter softwareudviklingsteori, og jeg har prøvet kræfter med backend, systemudvikling og firmwareudvikling.
        Det, der virkelig fanger mig, er at udforske nuancerne og konstruktionerne i forskellige programmeringssprog, hvilket giver mig indsigt i, hvad der er bedst egnet til forskellige problemer (i øjeblikket elsker jeg funktionel programmering).
        Denne blanding af teoretisk viden og praktiske færdigheder hjælper mig på min rejse i den dynamiske verden af software udvikling.
    }\\\\[-4pt]
    \href{tel:+4551164199}{\faIcon{mobile-alt}~+45 51 16 41 99} \\\\[-4pt]%[8pt] 
    \href{https://github.com/t-lohse}{\faIcon{github}~\footnotesize\faIcon{at}\normalsize\texttt{t-lohse}} \\\\[-4pt]%[8pt]
    \href{https://gitlab.com/tlohse}{\faIcon{gitlab}~\footnotesize\faIcon{at}\normalsize\texttt{tlohse}} \\\\[-4pt]%[8pt]
    \href{mailto:mail@tlohse.dk}{\faIcon{envelope}~mail\normalsize\MVAt tlohse.dk} \\\\[-4pt]%[8pt]
    \href{https://linkedin.com/in/thomas-lohse}{\faIcon{linkedin}~thomas-lohse}\\\\[-4pt]
    \href{https://tlohse.dk}{\faIcon{link}~tlohse.dk}\\\\[-11pt]
    \bottomrule
\end{tabularx}%\\[8pt]
%\par\noindent\rule{\textwidth}{1pt}
\section*{Tekniske Komptencer}
\secsep
\vspace{1em}
%\setlength{\tabcolsep}{7pt}
\newcolumntype{Y}{>{\centering\arraybackslash}X}
\iffalse
\begin{tabularx}{\textwidth}{YYY}
    \centering
    \large\textbf{Software Development} & \textbf{git} & \textbf{Project Development} \\
    \normalsize Since 2017              & Since 2018         & Since 2017                         
\end{tabularx}%
\fi

%\vspace{1em}\hrule width2.5cm\vspace{1em}

%\begin{tabularx}{\linewidth}{l|l|l}
\begin{center}
    \begin{tabular}{l|l|l}
        %\textbf{Known Programming Languages:} & \textbf{Known Tools:} \\
        {\begin{tabular}[t]{lll}
            \textbf{Language} & \textbf{Since} & \textbf{Level} \\
            Haskell & 2023 & 8/10 \\
            Kotlin  & 2022 & 8/10 \\
            Java    & 2022 & 8/10 \\
            Rust    & 2021 & 9/10 \\
            Bash    & 2020 & 7/10 \\
            C/C++   & 2020 & 7/10 \\
            C\#     & 2017 & 8/10
        \end{tabular}}
        &
        {\begin{tabular}[t]{lll}
            \textbf{Tool} & \textbf{Since} & \textbf{Level} \\
            Cabal  & 2023 & 7/10 \\
            Docker & 2021 & 7/10 \\
            Cargo  & 2021 & 8/10 \\
            Linux  & 2020 & 8/10 \\
            GitHub & 2018 & 9/10 \\
            Git    & 2018 & 9/10 \\
            .NET   & 2018 & 7/10
        \end{tabular}}
        &
        {\begin{tabularx}{\linewidth}[t]{lll}
            \textbf{Additional} & \textbf{Since} & \textbf{Level} \\
            CI/CD       & 2022 & 8/10 \\
            SCRUM       & 2022 & 8/10 \\
            REST API    & 2021 & 7/10 \\
            Embedded    & 2019 & 7/10 \\
            Projekter   & 2017 & 9/10 \\
            Programming & 2017 & 9/10 \\ 
            %BUZZWORD PLS
        \end{tabularx}}
    \end{tabular}          
\end{center}

\newcommand{\p}[1]{\textbf{#1}\mbox{}\newline}
\def\n{\\\\}
\section*{Uddannelse}
\secsep
\begin{longtabu} to \textwidth {r|X}
    2023--Now & \p{Universitet (Master)}
    Studenrende ved Aalborg Universitet på Software Kandidatuddannelsen. Det følgende beskriver projekterne på hvert semester (* er det nuværende):
    \begin{itemize}[leftmargin=4em]
        \item[\textbf{1.}] Udviklede en onling læringsplatform til programmingsintroduktion, bestående af flere forskellige services, som en REST API, frontend, og en separat service til at afvikle kode.
            Projektet blev deployed med Docker Swarm, og skrevet i TypeScript.
        \item[\textbf{*2.}] TBA
    \end{itemize}
    \\
    2020--2023 & \p{Universitet (Bachelor)}
    Studenrende ved Aalborg Universitet på Software Bacheloruddannelsen. Det følgende beskriver projekterne på hvert semester:
    \begin{itemize}[leftmargin=4em]
        \item[\textbf{1.}] Udviklede et vagtplanlægningssystem for Siemens Gamesas produktionsarbejderhold (Skrevet i C).
        \item[\textbf{2.}] Implementerede Signal Protokollen i et IoT-miljø (Skrevet i JavaScript).
        \item[\textbf{3.}] Udviklede et program til bedre håndtering af Siemens Gamesas vindturbinevingers lokation og produktion (Skrevet i C\#).
        \item[\textbf{4.}] Udviklede et programmingssprog som erstatning af Bash shell scripting (Skrevet i C\#)
        \item[\textbf{5.}] Et multiproject, hvor seks grupper skulle samarbejde i den samme kodebase (Skrevet i Rust).
        \item[\textbf{6.}] Udviklede et model learning værktøj til ar reverse engineer kildekoden fra PLCer i ladder logic (Skrevet i Java og C\#).
    \end{itemize}
    \\
    2017--2020 & \p{Gymnaisum} 
    Elev på Aalborg Techcollege (AATG), hvor jeg tog den tekniske studentereksamen (HTX),
    med profilfagende \textit{Kommunikation \& IT} A og \textit{Programmering} B,
    samt teknikfag i \textit{Elektronisk Udvikling og Produktion} A.
\end{longtabu}
    \iffalse%
    \\\\
    2016--2017 & \textbf{Continuation School}\\lign=right, labelindent=!, leftmargin=6em, itemindent=!, labelsep=1em, labelwidth=!]
    &   Attendee at Ingstrup Efterskole in grade nine.
    \\\\
    2007--2016 & \textbf{Public School}\\
    &   Attendee at the public school in 9310 Vodskov (Vodskov Skole) from grade one to grade eight.
    \fi%

\section*{Beskæftigelse}
\secsep
\begin{longtabu} to \textwidth {r|X}
    Efterår 2023 & \p{Product Owner for 5. semester Software multiproject}
    Jeg var product ownser for multiprojektet på 5. semester Software, bestående af seks grupper med cirka seks medlemmer i hver gruppe. Projektet var på 15 ECTS-point.
    Mine ansvarsområder inkluderede at lave opgaver der skulle implementeres af dem, besvarelse af spørgsmål vedrørende kodebasen og opgaverne, samt generelt at fungere som vejleder i projektet og som product owner.
    \n
    Efterår 2022 & \p{Undervisningsassistent}
    Jeg var undervisningsassistent på Imperativ Programmeringkurset på 1. semester af Software og Datalogi på AAU.
    Dette er det første programmeringskursus, man har på uddannelserne. Jeg skulle hjælpe studerende under øvelsessessioner, give feedback på deres afleveringsopgaver og samarbejde med kursuslæreren vedrørende øvelserne, platformen og afleveringerne.
    \n
    2022--2024 & \p{Aalborg Universitet, DEIS}
    Jeg arbejder som studentermedarbejder hos DEIS på \href{https://github.com/Ecdar}{Ecdar}-projektet, hvor jeg primært udvikler i Rust i backenden til model checking (\href{https://github.com/Ecdar/Reveaal}{Reveaal}), men bidrager også til de andre dele af projektet.
    \n
    2022 & \p{RTX A/S}
    Jeg arbejdede som studentermedhjælper hos RTX A/S i Nørresundby med både hardware- og softwareopgaver, som omfattede lodning og samling af udstyr samt udvikling af platforme til overvågning og test af større enheder/komponenter.
    \n
    2018--2019 & \p{Føtex Nørresundby}
    I løbet af min ansættelse hos Føtex Nørresundby havde jeg forskellige roller:
    \begin{itemize}[leftmargin=11em]%\setlength\itemsep{0em}
        \item[\textbf{Servicemedarbejder}] Min første kontrakt var under stillingen 
            Servicemedarbejder, som havde et udvalg af opgaver, hvor den primære var at
            opretholde pantflaskemaskinerne.
        \item[\textbf{Kasseassistent}] Omkring et halvt år efter min ansættelse, bliver
            jeg tilbudt en oplæring og ny tilhørende stilling som Kasse Assistent.
        \item[\textbf{Bake-Off Sal}] Da jeg fylder 18 år, og min kontrakt opsiges,
            tilbyder de mig en ny kontrakt med stillingen Bake-Off Salsperson, som var min stilling
            op til min opsigelse.
    \end{itemize}
\end{longtabu}

\section*{Anden Erfaring}
\secsep
\begin{longtabu} to \textwidth {r|X}
Efterår 2023 & \p{Tutorkoordinator}
Jeg arbejdede som Tutorkoordinator for nye studerende i 2023, der startede på Software-studiet på Aalborg Universitet.
Jeg var ansvarlig for at planlægge og organisere tutorplanerne og sikre, at de var på rette spor med planlægning, fundraising og afvikling af deres arrangementer, sammen med nogle arrangementer, jeg også skulle planlægge.
Jeg var også ansvarlig for at håndtere organisationens økonomi, hvilket inkluderede afsendelse og betaling af fakturaer samt budgettering.
\n
Efterår 2022 & \p{Tutorplanlægger}
Jeg arbejdede som Tutorplanlægger for nye studerende i 2022, der startede på Software-studiet på Aalborg Universitet.
Jeg var ansvarlig for at planlægge, indsamle midler og afvikle arrangementet, hvor jeg skulle koordinere en håndfuld tutorer for at gennemføre arrangementet korrekt.
\n
Efterår 2021 & \p{Tutor}
Jeg arbejdede frivilligt som tutor for nye studerende i 2021, der startede på Software-studiet på Aalborg Universitet.
\n
2021--nu & \p{UNF Spiludviklingslejr}
Jeg er frivillig ved UNFs Spiludviklingslejr, hvor jeg har haft følgende roller:
\begin{itemize}[leftmargin=4em]
    \item[2024] \p{Fordreagshjælper} Assisterede med at finde og skaffe foredragsholdere til at præsentere for lejrdeltagerne samt dommere til at evaluere de spil, der bliver lavet i løbet af lejren.
    \item[2024] \p{Tryk- og Dagsseddelhjælper} Assisterede med at skaffe tilladelser fra spilstudier til kunstværkdisplays og skabte morgenavisen for lejrdeltagerne.
    \item[2023] \p{Tillidsperson} En af de få kontaktpersoner vedrørende lejrdeltagernes velfærd.
    \item[2022] \p{Teknisk Ansvarlig} Ansvarlig for opsætningen af alt det teknsike
            udstyrtil campen, samt introduktionen af git for deltagerne, og håndtering
            af alle problemer de måtte have med al det tekniske.
    \item[2021] \p{Programmingsassistent} Assisterer programmeringslæreren, samt
            hjælp med deltagernes programmeringsrelateret problemer.
    \item[2021] \p{Logistisk Ansvarlig} Ansvarlig for håndtering af logistiken
            vedrørende de frivilliges anmodninger om varer, samt anskaffelsen.
\end{itemize}
\end{longtabu}
\end{document}


