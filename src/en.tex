\documentclass[a4paper]{report}
\usepackage[utf8]{inputenc}
\usepackage{anysize}
\usepackage[a4paper,margin=7em]{geometry}
\usepackage{float}
\usepackage{graphicx} % Required for figures
\usepackage{booktabs}
\usepackage{hyperref}
\usepackage{fontawesome5}
\usepackage{multirow}
\usepackage{array}
\usepackage{lipsum}
\usepackage[export]{adjustbox}
\usepackage{marvosym}
\usepackage{tabularx}
%\usepackage{ltablex}
\usepackage{tabu}
\usepackage{longtable}[=v4.13]
\usepackage{enumitem}
\setitemize{itemsep=2pt, topsep=2pt, align=right, labelindent=\parindent, leftmargin=!, itemindent=!, labelsep=.5em, labelwidth=!}

\def\secsep{\hrule width5cm}
%\tabulinesep=0pt
%\extrarowsep=-3pt
%\def\secsep{\hrule}
\title{\bfseries\Huge CV}
\begin{document}
%\section*{\Huge CV}\subsection*{Thomas Krogh Lohse}
\section*{\centering \Huge CV}\subsection*{\centering Thomas Krogh Lohse}

\begin{tabularx}{\textwidth}{lX}
    \toprule%\\[-10pt]
    \href{https://maps.app.goo.gl/mtFWbUVz1f8x7saS8}{\faIcon{map-marker-alt}~Aalborg Øst 9220, Denmark}& 
    \multirow[t]{7}{=}{
        I am a Software Engineering Master (MSc) student at Aalborg University (AAU)% with a fervor for backend and system development.
        %Since the age of 16, my journey in programming has been marked by a commitment to independent learning, particularly in the realms of software development theory.
        %I specialize in backend development, with practical experience in firmware development and task automation using shell-scripts.
        %Beyond this, my passion extends to exploring various programming languages and their intricacies.
        %This blend of theoretical knowledge and hands-on skills positions me to excel in the intricate landscape of software engineering.
        , where my fascination with programming has taken me, on my journey of independent learning since the age of 16.
        My interests span software development theory, and I've dabbled in backend and system development, gaining hands-on experience in firmware development and task automation using shell-scripts.
        What truly captivates me is exploring the nuances and constructs of different programming languages, giving me insight in what is best suitable for different problems (Currently love functional programming).
        This mix of theoretical knowledge and practical skills aids my journey in the dynamic world of software engineering.
        %This mix of theoretical knowledge and practical skills sets the stage for my journey in the dynamic world of software engineering.
        %I am a Software Engineering Master's student at Aalborg University (AAU), with a passion for theoretical and practical aspects of programming.
        %Since the age of 16, I've actively engaged in independent learning, focusing on software development theory.
        %My expertise lies in backend and/or system development, with hands-on experience in firmware development, task automation using shell-scripts, and language parsing.
    }\\\\[-4pt]
    \href{tel:+4551164199}{\faIcon{mobile-alt}~+45 51 16 41 99} \\\\[-4pt]%[8pt] 
    \href{https://github.com/t-lohse}{\faIcon{github}~\footnotesize\faIcon{at}\normalsize\texttt{t-lohse}} \\\\[-4pt]%[8pt]
    \href{https://gitlab.com/tlohse}{\faIcon{gitlab}~\footnotesize\faIcon{at}\normalsize\texttt{tlohse}} \\\\[-4pt]%[8pt]
    \href{mailto:mail@tlohse.dk}{\faIcon{envelope}~mail\normalsize\MVAt tlohse.dk} \\\\[-4pt]%[8pt]
    \href{https://linkedin.com/in/thomas-lohse}{\faIcon{linkedin}~thomas-lohse}\\\\[-4pt]
    \href{https://tlohse.dk}{\faIcon{link}~tlohse.dk}\\\\[-11pt]
    \bottomrule
\end{tabularx}%\\[8pt]
%\par\noindent\rule{\textwidth}{1pt}
\section*{Technical Competencies}
\secsep
\vspace{1em}
%\setlength{\tabcolsep}{7pt}
\newcolumntype{Y}{>{\centering\arraybackslash}X}
\iffalse
\begin{tabularx}{\textwidth}{YYY}
    \centering
    \large\textbf{Software Development} & \textbf{git} & \textbf{Project Development} \\
    \normalsize Since 2017              & Since 2018         & Since 2017                         
\end{tabularx}%
\fi

%\vspace{1em}\hrule width2.5cm\vspace{1em}

%\begin{tabularx}{\linewidth}{l|l|l}
\begin{center}
    \begin{tabular}{l|l|l}
        %\textbf{Known Programming Languages:} & \textbf{Known Tools:} \\
        {\begin{tabular}[t]{lll}
            \textbf{Language} & \textbf{Since} & \textbf{Level} \\
            Haskell & 2023 & 8/10 \\
            Kotlin  & 2022 & 8/10 \\
            Java    & 2022 & 8/10 \\
            Rust    & 2021 & 9/10 \\
            Bash    & 2020 & 7/10 \\
            C/C++   & 2020 & 7/10 \\
            C\#     & 2017 & 8/10
        \end{tabular}}
        &
        {\begin{tabular}[t]{lll}
            \textbf{Tool} & \textbf{Since} & \textbf{Level} \\
            Cabal  & 2023 & 7/10 \\
            Docker & 2021 & 7/10 \\
            Cargo  & 2021 & 8/10 \\
            Linux  & 2020 & 8/10 \\
            GitHub & 2018 & 9/10 \\
            Git    & 2018 & 9/10 \\
            .NET   & 2018 & 7/10
        \end{tabular}}
        &
        {\begin{tabularx}{\linewidth}[t]{lll}
            \textbf{Additional} & \textbf{Since} & \textbf{Level} \\
            CI/CD       & 2022 & 8/10 \\
            SCRUM       & 2022 & 8/10 \\
            REST API    & 2021 & 7/10 \\
            Embedded    & 2019 & 7/10 \\
            Projects    & 2017 & 9/10 \\
            Programming & 2017 & 9/10 \\ 
            %BUZZWORD PLS
        \end{tabularx}}
    \end{tabular}          
\end{center}

\newcommand{\p}[1]{\textbf{#1}\mbox{}\newline}
\def\n{\\\\}
\section*{Education}
\secsep
\begin{longtabu} to \textwidth {r|X}
    2023--Now & \p{University (Master)}
    Attendee at Aalborg University, studying Software Engineer, Master. The following describes the project from each semester (* is the current):
    \begin{itemize}[leftmargin=4em]
        \item[\textbf{1st}] Engineered an online learning platform for introductionary programming, with multiple services, for example a REST API, frontend, and a seperate service for executing code. The project was deployed with Docker Swarm, and written in TypeScript.
        \item[\textbf{2nd}] Designed and implemented a protocol for ad-hoc dynamic mesh networks, and utilized it in an app for sharing users' location with selected ``friend users" in this network, by sending Bluetooth advertisements with the data. The project was developed for Android devices, and written in Kotlin.
        \item[\textbf{*3rd}] \textbf{(Master's preliminary thesis)} Will work with safety in C++ through static analysis.
    \end{itemize}
    \\
    2020--2023 & \p{University (Bachelor)}
    Attendee at Aalborg University, studying Software, Bachelor. The following describes the project from each semester:
    \begin{itemize}[leftmargin=4em]
        \item[\textbf{1st}] Developed a workscheduling system for the production teams at Siemens Gamesa (Written in C).
        \item[\textbf{2nd}] Implemented the Signal Protocol in an IoT environment (Written in JavaScript).
        \item[\textbf{3rd}] Developed a program for better handling of Siemens Gamesas turbine blades' location and production (Written in C\#).
        \item[\textbf{4th}] Developed a programming language as a replacement to the shell scripting language Bash (Written in C\#).
        \item[\textbf{5th}] A multiproject, where six groups collaborated on the same code base (Written in Rust).
        \item[\textbf{6th}] Developed a model learning tool to reverse engineer the source code from PLCs in ladder logic (Written in Java and C\#).
    \end{itemize}
    \\
    2017--2020 & \p{High School} 
    High school attendee for three years at Aalborg Techcollege (AATG), and recieved the Higher
    Techincal Exam (HTX), with a specialised Study Area in \textit{Communication \& IT} A,
    \textit{Programming} B, and tecnical study in \textit{Technical Science} A.
\end{longtabu}
    \iffalse%
    \\\\
    2016--2017 & \textbf{Continuation School}\\lign=right, labelindent=!, leftmargin=6em, itemindent=!, labelsep=1em, labelwidth=!]
    &   Attendee at Ingstrup Efterskole in grade nine.
    \\\\
    2007--2016 & \textbf{Public School}\\
    &   Attendee at the public school in 9310 Vodskov (Vodskov Skole) from grade one to grade eight.
    \fi%

\section*{Employment}
\secsep
\begin{longtabu} to \textwidth {r|X}
    Fall 2022 & \p{Teaching Assistant}
    I am a Teaching Assistant in the Programming Paradigms course on the 1st semester of the Masters in Software, Computer Science, and Data Science at AAU.
    The course primarily deals with functional programming in Haskell, where I help with teaching and the exercises.
    \n
    Fall 2023 & \p{Product Owner for 5th semester Software multiproject}
    I was the overall product owner for the multiproject on 5th semester Software, consisting of six groups with about six members in each group. The project was 15 ECTS points.
    My responsibility included creating the tasks that should be implemented by them, answering questions regarding the code base and tasks, and generally acting as a guidance on the project, and as a PO.
    \n
    Fall 2022 & \p{Teaching Assistant}
    I was a Teaching Assistant in the Imperative Programming course on the 1st semester of Software and Computer Science at AAU.
    This is the first programming course you have on the educations. I had to help students during exercise sessions, provide feedback on their hand-in assignments, and collaborate with the course lecturer regarding the exercises, platform, and hand-ins.
    \n
    2022--2024 & \p{Aalborg University, DEIS}
    I worked as a Student Software developer at DEIS on the \href{https://github.com/Ecdar}{Ecdar} project
    where I developed mainly in Rust in the backend engine for model checking
    (\href{https://github.com/Ecdar/Reveaal}{Reveaal}), but also contributed to the other parts of the project.
    \n
    2022 & \p{RTX A/S}
    I worked as a student assistant at RTX A/S in Nørresundby, with both hardware and software tasks,
    consisting of soldering og assembling equipment, along with developing platforms for monitoring and testing of larger devices/components.
\end{longtabu}
\iffalse
    \n
    2018--2019 & \p{Føtex Nørresundby}
    Over the time of my employment at Føtex Nørresundby, I had a selection of roles:
    \begin{itemize}[leftmargin=11em]%\setlength\itemsep{0em}
        \item[\textbf{Service employee}] My first contract was as a Service employee,
            with a variety of tasks, primarily the operation of the bottle recycle machine.
        \item[\textbf{Cashier Assistant}] About a half year after my employment,
            I was asked to be trained and reloacated to a new position, Cahsier Assistant.
        \item[\textbf{Bake-Off Sale}] When I turned 18 years old, and my contract
            terminated, I was offered a new position as Bake-Off Salesperson, which was my
            position up until my resignation.%
    \end{itemize}
\fi

\section*{Other Experience}
\secsep
\begin{longtabu} to \textwidth {r|X}
Fall 2023 & \p{Tutor Coordinator}
I volunteered as a tutor coordinator for the students in 2023, studying Software at Aalborg University.
I was responsible for planning and organizing the tutor planners, making sure that they were on track with planning, fundraising, and executing their events, along with a few events I had to plan as well.
I was also responsible for handling the finances of the orginazation, which included sending and paying invoices and budgetting.
\n
Fall 2022, 2024 & \p{Tutor Planner}
I volunteered as a tutor planner for the students in 2022 and 2024, studying Software at Aalborg University.
I was responsible for planning, fundraise, and execute the event, where I had to coordinate a handful of tutors to execute the event properly.
\n
    Fall 2021 & \p{Tutor}
        I volunteered as a tutor for the students in 2021, studying Software at Aalborg University.
\n
    2021--2024 & \p{UNF Game Development Camp}
    I have been a volunteer at UNFs Game Development Camp, where I have had the following (relevant) roles:
    \begin{itemize}[leftmargin=4em]
        %\item[2024] \p{Lecture Assistant} Asisting in finding and acquiring lecturers to present for the campers, along with the judges for evaluating the games that are made throughout the camp.
        %\item[2024] \p{Printing and Morning Paper Assistant} Assisting in acquiring permissions from game studios for artwork desiplys, and creating the morning paper for the campers.
        %\item[2023] \p{Trustee} One of the few contact persons regarding the weathfare of campers.
        \item[2022] \p{Technical Manager} In charge of making sure the campers'
            technical equipment is operational and set up correctly,
            along with introducing them to \texttt{git}, and managing
            the problems they might have with the technalities.
        \item[2021] \p{Programmings Assistant} Assisting the programming teacher, and
            helping the campers with any programming related issues they might have.
        %\item[2021] \p{Logistics Responsible} In charge of handling the logistical
        %    requests of the other volunteers, and acquiring said logistics.
\end{itemize}
\end{longtabu}
\iffalse
\section*{References}
\hrule width5cm \ \\
Provided upon request.
\fi
\end{document}

