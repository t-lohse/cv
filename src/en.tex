\documentclass[a4paper]{report}
\usepackage[utf8]{inputenc}
\usepackage{anysize}
\usepackage[a4paper,margin=7em]{geometry}
\usepackage{float}
\usepackage{graphicx} % Required for figures
\usepackage{booktabs}
\usepackage{hyperref}
\usepackage{fontawesome5}
\usepackage{multirow}
\usepackage{array}
\usepackage{lipsum}
\usepackage[export]{adjustbox}
\usepackage{marvosym}
\usepackage{tabularx}
\usepackage{multicol}
%\usepackage{ltablex}
\usepackage{tabu}
\usepackage{longtable}[=v4.13]
\usepackage{enumitem}
\setitemize{itemsep=2pt, topsep=2pt, align=right, labelindent=\parindent, leftmargin=!, itemindent=!, labelsep=.5em, labelwidth=!}

\def\secsep{\hrule width5cm}
%\tabulinesep=0pt
%\extrarowsep=-3pt
%\def\secsep{\hrule}
\title{\bfseries\Huge CV}
\begin{document}
%\section*{\Huge CV}\subsection*{Thomas Krogh Lohse}
\section*{\centering \Huge CV}\subsection*{\centering Thomas Krogh Lohse}

\begin{tabularx}{\textwidth}{lX}
    \toprule%\\[-10pt]
    \href{https://maps.app.goo.gl/mtFWbUVz1f8x7saS8}{\faIcon{map-marker-alt}~Aalborg Øst, Denmark}& 
    \multirow[t]{6}{=}{
    I'm a Master's student in Software Engineering (MSc) at Aalborg University (AAU).
    My programming journey began at 16, and I've since gained hands-on experience in many areas, from firmware development to REST APIs and programming language theory.
    I love programming language theory, especially type systems and static analysis, and enjoy exploring how different paradigms influence software design (currently love functional programming).
    I'm eager to use my technical skills, theoretical knowledge, and passion to solve real-world software challenges.
        %I am a Software Engineering Master (MSc) student at Aalborg University (AAU)% with a fervor for backend and system development.
        %, where my fascination with programming has taken me, on my journey of independent learning since the age of 16.
        %My interests span software development theory, and I've dabbled in backend and system development, gaining hands-on experience in firmware development and task automation using shell-scripts.
        %What truly captivates me is exploring the nuances and constructs of different programming languages, giving me insight in what is best suitable for different problems (Currently love functional programming).
        %This mix of theoretical knowledge and practical skills aids my journey in the dynamic world of software engineering.
    }\\\\[-4pt]
    \href{tel:+4551164199}{\faIcon{mobile-alt}~+45 51 16 41 99} \\\\[-5pt]%[8pt] 
    \href{https://github.com/t-lohse}{\faIcon{github}~\footnotesize\faIcon{at}\normalsize\texttt{t-lohse}} \\\\[-5pt]%[7pt]
    %\href{https://gitlab.com/tlohse}{\faIcon{gitlab}~\footnotesize\faIcon{at}\normalsize\texttt{tlohse}} \\\\[-5pt]%[8pt]
    \href{mailto:mail@tlohse.dk}{\faIcon{envelope}~mail\normalsize\MVAt tlohse.dk} \\\\[-5pt]%[8pt]
    \href{https://linkedin.com/in/thomas-lohse}{\faIcon{linkedin}~thomas-lohse}\\\\[-5pt]
    \href{https://tlohse.dk}{\faIcon{link}~tlohse.dk}\\\\[-12pt]
    \bottomrule
\end{tabularx}%\\[8pt]
%\par\noindent\rule{\textwidth}{1pt}
\subsection*{Selected Competencies}
\secsep
\vspace{-1em}
%\setlength{\tabcolsep}{7pt}
\newcolumntype{Y}{>{\centering\arraybackslash}X}
\newcommand{\subpart}[1]{\begin{itemize}[leftmargin=-2em, topsep=-.9em, parsep=0em]
    \item[]  {\scriptsize #1}
\end{itemize}}

\begin{multicols}{3}
    \centering
    \begin{itemize}[leftmargin=2em, topsep=-.5em, parsep=0em]
        \item Systems programming
            \subpart{Rust, C/C++}
        \item Functional Programming
            \subpart{Haskell, OCaml, Rust}
        \item Desktop Linux
            \subpart{Bash, Systemd, SSH}
        \item Deployment
            \subpart{Docker (Swarm), Docker Compose}
        \item CI/CD
            \subpart{Github Actions, Gitlab workflows}
        \item Embedded programming
            \subpart{C/C++, Arduino, firmware}
        \item Project management
            \subpart{Scrum, Git}
        \item MicroServices \& SOA
            \subpart{REST API, Docker Swarm}
        \item Language theory
            \subpart{Static analysis, type systems}
    \end{itemize}
\end{multicols}
%\vspace{1em}\hrule width2.5cm\vspace{1em}

\newcommand{\p}[1]{\textbf{#1}\mbox{}\newline}
\def\n{\\\\}
\subsection*{Education}
\secsep
\begin{longtabu} to \textwidth {r|X}
    2023--Now & \p{\href{https://studieordninger.aau.dk/2023/41/4304}{Master of Science (MSc) in Engineering (Software), Aalborg University (AAU)}}
    %Attendee at Aalborg University, studying Software Engineer, Master. The following describes the project from each semester (Currenlty on 4th semester):
    \vspace{-1em}
    \begin{itemize}[leftmargin=2em, topsep=-.5em, parsep=0em]
        \item Service oriented arcitechture using RESP APIs and scalable deployment through Docker Swarm.
        \item Protocol design and implementation for ad-hoc dynamic mesh networks, using Bluetooth.
        \item Researching language safety and static analysis, comparing C++ and Rust w.r.t language safety.
        %\item Engineered an online learning platform for introductionary programming, with multiple services, for example a REST API, frontend, and a seperate service for executing code. The project was deployed with Docker Swarm, and written in TypeScript.
        %\item Designed and implemented a protocol for ad-hoc dynamic mesh networks, and utilized it in an app for sharing users' location with selected ``friend users" in this network, by sending Bluetooth advertisements with the data. The project was developed for Android devices, and written in Kotlin.

        %\item  \textbf{(Master's preliminary thesis)} Will work with safety in C++ through static analysis.
    \end{itemize}
    \vspace{-.1em}
    \\
    %\vspace{-1em}
        &  \textbf{Master's thesis:} Safety in C++
        \begin{itemize}[leftmargin=2em, topsep=-.5em, parsep=0em]
            \item Implementing static analysis for C++ to represent ownership and discover aliasing at compile time.
            \item Analysis implemented as a Clang plugin in C++.
        \end{itemize}
    \\
    2020--2023 & \p{\href{https://studieordninger.aau.dk/2023/38/3925}{Bachelor of Science (BSc) in Engineering (Software), Aalborg University (AAU)}}
    %Attendee at Aalborg University, studying Software, Bachelor. The following describes the project from each semester:
    \vspace{-1em}
    \begin{itemize}[leftmargin=2em, topsep=-.5em, parsep=0em]
        
        %\item Worked with different programming languages and problems.

        \item Developed a workscheduling system for the production teams at Siemens Gamesa (Written in C).
        \item Implemented the Signal Protocol in an IoT environment (Written in JavaScript).
        \item Developed a program for better handling of Siemens Gamesas turbine blades' location and production (Written in C\#).
        \item Developed a programming language as a replacement to the shell scripting language Bash (Written in C\#).
        \item A multiproject, where six groups collaborated on the same code base (Written in Rust).
    \end{itemize}
    \vspace{-.1em}
    \\
        &  \textbf{Bacelor's thesis:} Reverse Engineering of PLC code
        \begin{itemize}[leftmargin=2em, topsep=-.5em, parsep=0em]
            \item Developed a model learning tool to reverse engineer the source code from PLCs in ladder logic
            \item Written in Java and C\#.
        \end{itemize}
    %\\
    %2017--2020 & \p{High School} 
    %High school attendee for three years at Aalborg Techcollege (AATG), and recieved the Higher
    %Techincal Exam (HTX), with a specialised Study Area in \textit{Communication \& IT} A,
    %\textit{Programming} B, and tecnical study in \textit{Technical Science} A.
\end{longtabu}
    \iffalse%
    \\\\
    2016--2017 & \textbf{Continuation School}\\lign=right, labelindent=!, leftmargin=6em, itemindent=!, labelsep=1em, labelwidth=!]
    &   Attendee at Ingstrup Efterskole in grade nine.
    \\\\
    2007--2016 & \textbf{Public School}\\
    &   Attendee at the public school in 9310 Vodskov (Vodskov Skole) from grade one to grade eight.
    \fi%

\subsection*{Work Experience}
\secsep
\begin{longtabu} to \textwidth {r|X}
    2025-Now & \p{Trifork}
    Student developer at Trifork, working with real-time programming and data processing.
    \n
    2024-Now & \p{Aalborg University}
    Student developer at Aalborg University, working on a tool written in Haskell to parse, interpret, and analyze RISC-V assembler code.
    \n
    Fall 2024 & \p{Teaching Assistant}
    Teaching Assistant in the Programming Paradigms course on the 1st semester of the Master's program. % in Software, Computer Science, and Data Science at AAU.
    %The course primarily deals with functional programming in Haskell, where I help with teaching and the exercises.
    \n
    Fall 2023 & \p{Product Owner for 5th semester Software multiproject}
    Product owner for the multiproject on 5th semester Software, consisting of six groups with about six members in each group. %The project was 15 ECTS points.
    %Responsibility included creating the tasks, answering questions regarding the code base and tasks, and acting as a guidance on the project, and as a PO.
    \n
    Fall 2022 & \p{Teaching Assistant}
    Teaching Assistant in the Imperative Programming course on the 1st semester of the Bachelor's prgram. %  Software and Computer Science at AAU.
    %This is the first programming course you have on the educations. I had to help students during exercise sessions, provide feedback on their hand-in assignments, and collaborate with the course lecturer regarding the exercises, platform, and hand-ins.
    \n
    2022--2024 & \p{Aalborg University, DEIS}
    Student Software developer at DEIS on the \href{https://github.com/Ecdar}{Ecdar} project.
    Developed mainly in Rust in the backend engine for model checking
    (\href{https://github.com/Ecdar/Reveaal}{Reveaal}), but also contributed to the other parts of the project.
    \n
    2022 & \p{RTX A/S}
    Student assistant at RTX A/S in Nørresundby, with both hardware and software tasks.
    %consisting of soldering og assembling equipment, along with developing platforms for monitoring and testing of larger devices/components.
\end{longtabu}
\iffalse
    \n
    2018--2019 & \p{Føtex Nørresundby}
    Over the time of my employment at Føtex Nørresundby, I had a selection of roles:
    \begin{itemize}[leftmargin=11em]%\setlength\itemsep{0em}
        \item[\textbf{Service employee}] My first contract was as a Service employee,
            with a variety of tasks, primarily the operation of the bottle recycle machine.
        \item[\textbf{Cashier Assistant}] About a half year after my employment,
            I was asked to be trained and reloacated to a new position, Cahsier Assistant.
        \item[\textbf{Bake-Off Sale}] When I turned 18 years old, and my contract
            terminated, I was offered a new position as Bake-Off Salesperson, which was my
            position up until my resignation.%
    \end{itemize}
\fi

\subsection*{Other Experience}
\secsep
\begin{longtabu} to \textwidth {r|X}
2021-2024 & \p{Tutor at AAU}
    I've participated in the tutor corps for the Software bachelor program throughout my studies, every year.
    In 2021, I was a normal tutor.
    In 2022 and 2024, I was a tutor planner, where I planned, fundraised, and executed social events for the new students.
    In 2023, I was the tutor coordinator, managing the corps and its funds, along with the same responsibility as a planner.
\n
    2021--2024 & \p{UNF Game Development Camp}
    I've volunteered at UNFs Game Development Camp.
    During these camps, I've had a few different responsibilities, such as teaching assistant for programming, and technical manager.
    %\begin{itemize}[leftmargin=4em]
    %    %\item[2024] \p{Lecture Assistant} Asisting in finding and acquiring lecturers to present for the campers, along with the judges for evaluating the games that are made throughout the camp.
    %    %\item[2024] \p{Printing and Morning Paper Assistant} Assisting in acquiring permissions from game studios for artwork desiplys, and creating the morning paper for the campers.
    %    %\item[2023] \p{Trustee} One of the few contact persons regarding the weathfare of campers.
    %    \item[2022] \p{Technical Manager} In charge of making sure the campers'
    %        technical equipment is operational and set up correctly,
    %        along with introducing them to \texttt{git}, and managing
    %        the problems they might have with the technalities.
    %    \item[2021] \p{Programmings Assistant} Assisting the programming teacher, and
    %        helping the campers with any programming related issues they might have.
    %    %\item[2021] \p{Logistics Responsible} In charge of handling the logistical
    %    %    requests of the other volunteers, and acquiring said logistics.
    %\end{itemize}
\end{longtabu}
\end{document}

