\documentclass[a4paper]{report}
\usepackage[utf8]{inputenc}
\usepackage{anysize}
\usepackage[a4paper,margin=7em]{geometry}
\usepackage{float}
\usepackage{graphicx} % Required for figures
\usepackage{booktabs}
\usepackage{hyperref}
\usepackage{fontawesome5}
\usepackage{multirow}
\usepackage{array}
\usepackage{lipsum}
\usepackage[export]{adjustbox}
\usepackage{marvosym}
\usepackage{tabularx}
\usepackage{multicol}
\usepackage{tabu}
\usepackage{longtable}[=v4.13]
\usepackage{enumitem}
\setitemize{itemsep=2pt, topsep=2pt, align=right, labelindent=\parindent, leftmargin=!, itemindent=!, labelsep=.5em, labelwidth=!}

\def\secsep{\hrule width5cm}
\title{\bfseries\Huge CV}
\begin{document}
\section*{\centering \Huge CV}\subsection*{\centering Thomas Krogh Lohse}

\begin{tabularx}{\textwidth}{lX}
    \toprule%\\[-10pt]
    \href{https://maps.app.goo.gl/mtFWbUVz1f8x7saS8}{\faIcon{map-marker-alt}~Aalborg Øst, Denmark}& 
    \multirow[t]{6}{=}{
    I'm a Master's student in Software Engineering (MSc) at Aalborg University (AAU).
    My programming journey began at 16, and I've since gained hands-on experience in many areas, from firmware development to REST APIs and programming language theory.
    I love programming language theory, especially type systems and static analysis, and enjoy exploring how different paradigms influence software design (currently love functional programming).
    I'm eager to use my technical skills, theoretical knowledge, and passion to solve real-world software challenges.
    }\\\\[-4pt]
    \href{tel:+4551164199}{\faIcon{mobile-alt}~+45 51 16 41 99} \\\\[-5pt]%[8pt] 
    \href{https://github.com/t-lohse}{\faIcon{github}~\footnotesize\faIcon{at}\normalsize\texttt{t-lohse}} \\\\[-5pt]%[7pt]
    \href{mailto:mail@tlohse.dk}{\faIcon{envelope}~mail\normalsize\MVAt tlohse.dk} \\\\[-5pt]%[8pt]
    \href{https://linkedin.com/in/thomas-lohse}{\faIcon{linkedin}~thomas-lohse}\\\\[-5pt]
    \href{https://tlohse.dk}{\faIcon{link}~\texttt{https://tlohse.dk}}\\\\[-12pt]
    \bottomrule
\end{tabularx}%\\[8pt]
\subsection*{Selected Competencies}
\secsep
\vspace{-1em}
\newcolumntype{Y}{>{\centering\arraybackslash}X}
\newcommand{\subpart}[1]{\begin{itemize}[leftmargin=-2em, topsep=-.9em, parsep=0em]
    \item[]  {\scriptsize #1}
\end{itemize}}

\begin{multicols}{3}
    \centering
    \begin{itemize}[leftmargin=2em, topsep=-.5em, parsep=0em]
        \item Systems programming
            \subpart{Rust, C/C++}
        \item Functional Programming
            \subpart{Haskell, OCaml, Rust}
        \item Desktop Linux
            \subpart{Bash, Systemd, SSH}
        \item Deployment
            \subpart{Docker (Swarm), Docker Compose}
        \item CI/CD
            \subpart{Github Actions, Gitlab workflows}
        \item Embedded programming
            \subpart{C/C++, Arduino, firmware}
        \item Project management
            \subpart{Scrum, Git}
        \item MicroServices \& SOA
            \subpart{REST API, Docker Swarm}
        \item Language theory
            \subpart{Static analysis, type systems}
    \end{itemize}
\end{multicols}

\newcommand{\p}[1]{\textbf{#1}\mbox{}\newline}
\def\n{\\\\}
\subsection*{Education}
\secsep
\begin{longtabu} to \textwidth {r|X}
    2023--Now & \p{\href{https://studieordninger.aau.dk/2023/41/4304}{Master of Science (MSc) in Engineering (Software), Aalborg University (AAU)}}
    \vspace{-1em}
    \begin{itemize}[leftmargin=2em, topsep=-.5em, parsep=0em]
        \item Designed, develped, and deployed larger, multi-service architectured systems.
        \item Low-level comunication protocl design and implementation for ad-hoc dynamic mesh networks.
        \item Language safety and static analysis research and implementation.
    \end{itemize}
    \vspace{-.1em}
    \\
        &  \textbf{Master's thesis:} \textit{No title yet}%Safety in C++
        \begin{itemize}[leftmargin=2em, topsep=-.5em, parsep=0em]
            \item Implementing static analysis for C++ to track runtime ownership and aliasing, at compile time.
            \item Analysis implemented as a Clang plugin in C++.
        \end{itemize}
    \\
    2020--2023 & \p{\href{https://studieordninger.aau.dk/2023/38/3925}{Bachelor of Science (BSc) in Engineering (Software), Aalborg University (AAU)}}
    \vspace{-1em}
    \begin{itemize}[leftmargin=2em, topsep=-.5em, parsep=0em]
        \item Advanced design and development of software systems.
        \item Engagement in several projects, including multi-team collaborations, using various agile development methods.
        \item Semester projects in various areas of software engineering, from web development to firmware to language implementation.
        \item Gained theoretical knowledge and practical use of it through exercises and the semester project.
    \end{itemize}
    \vspace{-.1em}
    \\
        %&  \href{https://github.com/cs-23-sw-6-12}{\textbf{Bacelor's thesis:} \textit{``Reverse Engineering PLCs into Ladder Programs: A Model Learning Approach''}}
        &  \textbf{Bacelor's thesis:} \textit{``Reverse Engineering PLCs into Ladder Programs: A Model Learning Approach''}
        \begin{itemize}[leftmargin=2em, topsep=-.5em,  parsep=0em]
            \item Develped a system that, using model learning, can reverse engineer (and potentially optimize) the running code on PLCs in ladder logic.
            \item Tested and benchmarked on different alogrithms on real hardware, determining the best suited algorithm for the task.
            \item Project consists of multiple services, written in Java and C\#.
        \end{itemize}
\end{longtabu}

\pagebreak
\subsection*{Work Experience}
\secsep
\begin{longtabu} to \textwidth {r|X}
    2025-Now & \p{Trifork}
    Student developer at \href{https://trifork.com/}{Trifork}, working with real-time programming and data processing.
    \n
    2024-Now & \p{Aalborg University}
    Student developer at Aalborg University, working on a tool written in Haskell to parse, interpret, and analyze RISC-V assembler code.
    \n
    Fall 2024 & \p{Teaching Assistant}
    Teaching Assistant in the Programming Paradigms course on the 1st semester of the Master's program. % in Software, Computer Science, and Data Science at AAU.
    \n
    Fall 2023 & \p{Product Owner for 5th semester Software multiproject}
    Product owner for the multiproject on 5th semester Software, consisting of six groups with about six members in each group. %The project was 15 ECTS points.
    \n
    Fall 2022 & \p{Teaching Assistant}
    Teaching Assistant in the Imperative Programming course on the 1st semester of the Bachelor's prgram. %  Software and Computer Science at AAU.
    \n
    2022--2024 & \p{Aalborg University, DEIS}
    Student Software developer at DEIS on the \href{https://github.com/Ecdar}{Ecdar} project.
    Developed mainly in Rust in the backend model checking engine (\href{https://github.com/Ecdar/Reveaal}{Reveaal}), but also contributed to the other parts of the project.
    \n
    2022 & \p{RTX A/S}
    Student assistant at RTX A/S in Nørresundby, with both hardware and software tasks.
\end{longtabu}

\subsection*{Other Experience}
\secsep
\begin{longtabu} to \textwidth {r|X}
2021-2024 & \p{Tutor at AAU}
    I've participated in the tutor corps for the Software bachelor program throughout my studies, every year.
    In 2021, I was a normal tutor.
    In 2022 and 2024, I was a tutor planner, where I planned, fundraised, and executed social events for the new students.
    In 2023, I was the tutor coordinator, managing the corps and its funds, along with the same responsibility as a planner.
\n
    2021--2024 & \p{UNF Game Development Camp}
    I've volunteered at UNFs Game Development Camp.
    During these camps, I've had a few different responsibilities, such as teaching assistant for programming, and technical manager.
\end{longtabu}

\subsection*{About me}
\secsep
\setlength{\parskip}{1ex}
In my sparetime, I enjoy staying active through lifting weights or bike cycling, which helps me clear my mind and stay focused, fit, and calm.
It’s a great way to challenge myself and set new goals, both physically and mentally.\par

\noindent
I also love watching different movies, series, or playing story-driven games, it’s a great way to relax and get lost and invested in different stories.

\noindent
Additionally, I enjoy discovering and trying different programming language concepts, primarily in the functional programming realm.
I like to do small coding exercises, and the occasional side project, it helps me stay active development-wise
\end{document}

